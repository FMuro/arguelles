% Arguello v1.2.0
% Copyright (c) 2020 Michele Piazzai. Contact: michele.piazzai@uc3m.es
% This work is released under the MIT License and is free to use, copy, modify,
% merge, publish, distribute, sublicense, and/or sell. See LICENSE for details.

\documentclass{beamer}
\usepackage[utf8]{inputenc}
\usepackage[T1]{fontenc}

\usetheme{Arguello}

\title{Argüello}
\subtitle{Simple, typographic beamer theme}
\date{}
\author{Place Holder}
\institute{University of \TeX\par\email{username@domain.com}}

\begin{document}

\frame[plain]{\titlepage
      \begin{tikzpicture}[remember picture,overlay]
            \node[xshift=-3cm,yshift=1.65cm] at (current page.south east) {\includegraphics[scale=.04]{logo_us_blanco}};
      \end{tikzpicture}
}

\Section{Demo}

\begin{frame}
      \frametitle{A frame with title and subtitle}
      \framesubtitle{The subtitle goes here}
      Lorem ipsum dolor sit amet, consectetur adipiscing elit, sed do eiusmod tempor incididunt ut labore et dolore magna aliqua.\par
      \vfill
      Ordered list:
      \begin{enumerate}
            \item First item
                  \begin{enumerate}
                        \item 1st item 2nd level
                              \begin{enumerate}
                                    \item 1st item 3rd level
                                    \item 2nd item 3rd level
                              \end{enumerate}
                        \item 2nd item 2nd level
                  \end{enumerate}
            \item Second item
            \item Third item
      \end{enumerate}
      \vfill
      Unordered list:
      \begin{itemize}
            \item First level
                  \begin{itemize}
                        \item Second level
                              \begin{itemize}
                                    \item Third level
                              \end{itemize}
                  \end{itemize}
      \end{itemize}
\end{frame}

\begin{frame}
      \frametitle{A frame with title only}
      \begin{theorem}
            \[e^{i\pi}+1=0\]
            \begin{proof}
                  \begin{equation*}
                        e^{iz}=\cos{z}+i\sin{z}
                  \end{equation*}
                  \center{therefore}
                  \begin{align*}
                        e^{i\pi}+1 & \null=\cos\pi+i\sin\pi+1 \\
                                   & \null=-1+i\times0+1      \\
                                   & \null=0
                  \end{align*}
            \end{proof}
      \end{theorem}
\end{frame}

\begin{frame}[plain]
      \frametitle{A plain frame has no headline}
      \begin{table}
            \small
            \begin{tabular}{rl}
                  \ttfamily\textbackslash Alegreya              & \Alegreya Lorem ipsum dolor sit amet              \\
                  \ttfamily\textbackslash AlegreyaExtraBold     & \AlegreyaExtraBold Lorem ipsum dolor sit amet     \\
                  \ttfamily\textbackslash AlegreyaBlack         & \AlegreyaBlack Lorem ipsum dolor sit amet         \\
                  \ttfamily\textbackslash AlegreyaMedium        & \AlegreyaMedium Lorem ipsum dolor sit amet        \\
                  \ttfamily\textbackslash AlegreyaSans          & \AlegreyaSans Lorem ipsum dolor sit amet          \\
                  \ttfamily\textbackslash AlegreyaSansThin      & \AlegreyaSansThin Lorem ipsum dolor sit amet      \\
                  \ttfamily\textbackslash AlegreyaSansLight     & \AlegreyaSansLight Lorem ipsum dolor sit amet     \\
                  \ttfamily\textbackslash AlegreyaSansMedium    & \AlegreyaSansMedium Lorem ipsum dolor sit amet    \\
                  \ttfamily\textbackslash AlegreyaSansExtraBold & \AlegreyaSansExtraBold Lorem ipsum dolor sit amet \\
                  \ttfamily\textbackslash AlegreyaSansBlack     & \AlegreyaSansBlack Lorem ipsum dolor sit amet
            \end{tabular}
      \end{table}
      \vfill
      \begin{alert}{Alert!}
            A plain frame does not show the progress bar but it still appears in the progress bar of other frames unless it is placed after \texttt{\textbackslash ThankYou}.
      \end{alert}
\end{frame}

\begin{frame}[standout]
      \Large
      A \textbf{\itshape\scshape standout} frame can be used to focus attention
\end{frame}

\begin{frame}
      \frametitle{Acknowledgements}

      This beamer theme is based in the Argüelles theme, originally developed by Michele Piazzai under the MIT license:

      \bigskip

      \url{https://github.com/piazzai/Arguello}

\end{frame}

\ThankYou
\begin{frame}[plain,standout]
      In combination with \textit{plain},\par
      it makes a nice thank-you slide!
      \vfill\scalebox{4}{\faGithub}\par\bigskip
      \url{https://github.com/FMuro/Arguello}
\end{frame}

\end{document}
